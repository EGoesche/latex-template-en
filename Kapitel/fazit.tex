% %%%%%%%%%%%%%%%%%%%%%%%%%%%%%%%%%%%%%%%%%%%%%%%%%%%%%%
% In dieser Datei steht der Text fuer das gemeinsame 
% Schluss/Zusammenfassung/Fazit-Kapitel.
% %%%%%%%%%%%%%%%%%%%%%%%%%%%%%%%%%%%%%%%%%%%%%%%%%%%%%%

\section{Fazit}%
\label{sec:fazit} 

Dies hier ist ein Blindtext zum Testen von Textausgaben. Wer diesen Text liest, ist selbst
schuld. Der Text gibt lediglich den Grauwert der Schrift an. Ist das wirklich so? Ist es
gleichg"ultig ob ich schreibe: ``Dies ist ein Blindtext'' oder ``Huardest gefburn''? Kjift 
mitnichten! Ein Blindtext bietet mir wichtige Informationen. An ihm messe ich die Lesbarkeit 
einer Schrift, ihre Anmutung, wie harmonisch die Figuren zueinander stehen
und pr"ufe, wie breit oder schmal sie l"auft. Ein Blindtext sollte m"oglichst viele 
verschiedene Buchstaben enthalten und in der Originalsprache gesetzt sein. Er muss keinen
Sinn ergeben, sollte aber lesbar sein. Fremdsprachige Texte wie ``Lorem ipsum'' dienen nicht 
dem eigentlichen Zweck, da sie eine falsche Anmutung vermitteln.
